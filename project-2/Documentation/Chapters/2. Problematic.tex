\section{Problemática a resolver}

La empresa MiCompany S.A., requiere de un mecanismo basado en tecnologías de 
información (TI) que le permita capturar la opinión de los consumidores en
relación con los diferentes productos que tienen en el mercado.

En este sentido, le han solicitado a su empresa de soluciones de TI, que
implemente una estrategia que le permita recolectar lo que piensan los usuarios
de los productos. En la actualidad, la empresa cuenta con un catálogo de
cincuenta (50) productos. Básicamente, lo que se requiere es que el usuario
pueda expresar lo que piensa de uno o varios productos en particular, en no mas
de ciento cincuenta caracteres (150). Tenga en cuenta que, la única intención de
la empresa es que, posterior al proceso de recolección de datos, se pueda
aplicar técnicas de minería de texto sobre la información recolectada (p.ej, 
sentiment analysis, topic detection, etc) con el fin de extraer información de
cómo perciben los productos los clientes.

El director de TI de su empresa, ha decido para satisfacer la necesidad, implementar y 
desplegar una aplicación web para este proyecto. Al respecto, se ha definido que la 
aplicación debe contar con un proceso de registro de usuarios el cual captura: nombre, 
edad, ciudad, dirección, correo electrónico, entre otros datos relevantes (información 
sociodemográfica de los usuarios). De igual forma, la aplicación debe permitir la 
visualización de cada uno de los productos que tiene la compañía en el mercado y le debe 
permitir al usuario seleccionar el producto para que este pueda dar su opinión de este. 
Vale resaltar que un usuario, puede seleccionar uno o varios productos con el fin de dar 
su opinión de este.

Su área es la encargada de tanto del desarrollo como el despliegue de la aplicación web. 
Para esto se requiere que usted realice todo el proceso de diseño de la aplicación, la 
implementación, pruebas y puesta en producción de esta. Con respecto al proceso de 
despliegue, su empresa ha decidido que ésta se desplegada considerando una 
infraestructura de TI robusta y escalable para soportar la operación de la aplicación. Todo 
esto con el fin de garantizar como mínimo una disponibilidad de $99.5\%$. Tenga en cuenta 
que la aplicación debe desarrollarse que soporte de manera concurrente miles de
usuarios al igual que debe permitir escalar de manera horizontal.

En este sentido, desde la perspectiva del desarrollo de software a nivel de tecnologías, 
usted puede considerar cualquiera de los Manejadores de Contenido (CMS por sus siglas 
en inglés) disponibles en la actualidad (Drupal o Wordrpess), con el fin de desarrollar la 
aplicación para satisfacer la problemática planteada. Para efectos de la persistencia de 
datos, se debe considerar utilizar bases de datos propias que emplean este tipo de 
soluciones. 
Por otro lado, en los aspectos relacionados con el despliegue, se ha decidido que la 
aplicación web debe desplegarse utilizando un proveedor de computación en nube, 
utilizando el modelo de infraestructura como servicio (IasS). Tenga en cuenta que para 
desplegar la aplicación se requiere que usted considere elementos como balanceadores 
de carga para lograr un buen despliegue de la solución, de tal forma, que el balanceador 
de carga sea configurado para distribuir las peticiones entrantes entre diferentes 
servidores web que se tienen y los usuarios puedan tener acceso a la aplicación web 
desplegada. Es importante resaltar que, para acceder la aplicación usted debe tener un 
gestionar y conseguir un dominio público gratis de tal forma que la aplicación debe 
accederse a través de una URL, como, por ejemplo, \url{http://www.micompany.tk}.